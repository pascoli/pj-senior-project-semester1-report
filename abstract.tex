\begin{abstract}

Current desktop 3D printers use FDM (Fused Deposition Modeling) to build parts out of flat layers of extruded thermoplastics. The printed parts have poor mechanical properties because of the low strength of thermoplastics and because the flat layer geometry limits inter-layer adhesion in thin areas. A curved-layer CFRP (Carbon Fiber Reinforced Polymer) 3D printer is being developed to solve those two issues. With curved layers, the carbon fiber may be oriented to best suit the applied loading on any given part, and the layers may be designed for greater inter-layer adhesion. An FDM-compatible ABS-matrix CFRP filament was developed and shown to have promising mechanical properties, comparable to aluminum. A custom FDM extruder was designed and prototyped for mounting on an available FANUC industrial robot arm, which provides the six necessary degrees of freedom to print curved layers. Control electronics were assembled and will be programmed to control the custom extruder and take input signals from the FANUC robot controller. Composite finite element analysis software was acquired and will be used to optimize the print layer geometry. Future work includes building the custom extruder; refining the filament production and test methods; programming the robot and extruder controller; generating optimized layer geometries; and printing and testing the CFRP material. 

\end{abstract}
