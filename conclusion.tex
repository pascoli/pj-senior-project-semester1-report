\section{Conclusion}

Current desktop 3D printers use FDM (Fused Deposition Modeling) to build parts out of planar layers of extruded thermoplastics, typically ABS or PLA. The printed parts have poor mechanical properties because of the low strength of thermoplastics and because the planar layer geometry limits inter-layer adhesion in features with thin areas. A curved-layer CFRP 3D printer is under development to solve those two issues. With curved layers, the carbon fiber may be oriented to best suit the applied loading on any given part, and the layers may be designed for greater inter-layer adhesion. An FDM-compatible ABS-matrix CFRP filament was developed and shown to have promising mechanical properties, comparable to aluminum. A custom FDM extruder was designed and prototyped for mounting on an available FANUC industrial robot arm, which provides the six necessary degrees of freedom to print curved layers. Control electronics were assembled and will be programmed to control the custom extruder and take input signals from the FANUC robot controller. A composite-specific finite element analysis software package was acquired and will be used to optimize the print layer geometry. Future work includes manufacturing the custom extruder; refining the filament production and test methods to create a printable CFRP; programming the robot and extruder controller; generating optimized layer geometries; and printing and testing the CFRP material. 

